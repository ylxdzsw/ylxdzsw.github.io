\documentclass{manuscript}

\usepackage[hidelinks]{hyperref}
\usepackage{textcomp}
\usepackage{enumitem}
\usepackage{amsmath}
\usepackage{tikz}

\title{A Bayesianism Theory of Judging and Sentencing}
\author{Zhang Shiwei}
\date{March 2018}

\begin{document}
    \maketitle

    \begin{abstract}
        This paper presents a new theory of justice where one get punished for the mathematical expectation of the
        ``evilness'' of his behaviour, no matter what the actual results. This helps prevent people from taking risk,
        and encourages them to be a good Samaria.
    \end{abstract}

    \section{A taste of this theory}

    \begin{quotation}
        Joe wants to kill Paul and therefor on the eve of Paul's setting forth on a hike across the desert, Joe sneaks
        into Paul's room and replaces the water in his canteen with scentless and colorless poison. Karl also wants to
        kill Paul and therefor later the same evening he sneaks into Paul's room and drills a small hole in the bottom
        of Paul's canteen. Paul leaves the next morning without noticing the hole in his canteen. After two hours in the
        desert he decides that it is time to drink but by now the canteen is empty. Without other sources of water he dies
        of dehydration in the desert. Who is responsible for the death?
    \end{quotation}

    This is a famous story mentioned in \cite{fletcher1998basic}. In most theories, Joe is either innocent or a attempted
    murderer, because Paul didn't die by drinking the poison. However, in our theory, the answer is somewhat simple: both
    Joe and Karl are guilty as murderers.

    The reasoning is that, they should be responsible for their action, not for Paul's death. The punishment for their
    actions depend on the possible results given the condition and their knowleadge at that time. For example, let $S(X)$
    be the punishment one should get for the event $X$, Joe's punishment should be
    $$\begin{aligned}
        E(S(X)P(X|C, K)|C, K) &= S(X_k)P(X_k|C,K) + S(X_{nk})P(X_{nk}|C,K) \\
                              &= S(X_k)P(X_k \cap X_d|C,K) + S(X_{nk})P(X_{nk} \cap X_d|C,K) + {} \\
                              &\phantom{{}={}} S(X_k)P(X_k \cap X_{nd}|C,K) + S(X_{nk})P(X_{nk} \cap X_{nd}|C,K)
    \end{aligned}$$

    Where

    \begin{itemize}[label={}, itemsep=0.1em]
        \item $X_k$: ``Joe kill Paul (Paul die by poison)''
        \item $X_{nk}$: ``Joe doesn't kill Paul''
        \item $X_d$: ``Paul's canteen is drilled''
        \item $X_{nd}$: ``Paul's canteen isn't drilled''
        \item $K$: ``The knowleadge of Joe (the fact that he don't know Karl's plan)''
        \item $C$: ``The condition at that time (drinking poison will cause death, etc.)''
    \end{itemize}

    We can have the jury agree with a set of values mentioned in the equation by common sense.

    \begin{itemize}[label={}, itemsep=0.1em]
        \item $P(X_k|X_d,C,K) = 0$ (``Paul won't die for poison if his canteen is drilled.'')
        \item $P(X_k|X_{nd},C,K) = 0.9999$ (``Paul tends to die for poison if his canteen is not drilled.'')
        \item $P(X_d|C, K) = 0.0001$ (``How likely the canteen will be drilled in Joe's mind'')
        \item $S(X_k) = 1$: ``The punishment for killing a man''
        \item $S(X_{nk}) = 0$: ``The punishment for not killing a man''
    \end{itemize}

    With these values determined, we can continue our calculation of Joe's punishment.

    $$\begin{aligned}
        E(\text{\ldots}) &= S(X_k)P(X_k|X_d,C,K)P(X_d|C,K) + 0 + {} \\
                         &\phantom{{}={}} S(X_k)P(X_k|X_{nd},C,K)(1-P(X_d|C,K)) + 0 \\
                         &= 0.9999 \approx 1
    \end{aligned}$$

    Thus, Joe deserves 0.9999 of the punishment for killing a man. The same calculation can be applyed to Karl. One can
    think that there are 3 possible results for drilling Paul's canteen: killing him, not killing him, and extending his
    life for 2 hours since otherwise he will die quicker for poison. The possibility and thus the contribution to the
    final expectation of the third result is obviours and it's easy to get to the same conclusion that Karl should also
    receive 0.9999 of the punishment of killing a man.

    The equation can still be expanded if necessary. For example, Paul may or may not notice the hole, and this is a
    factor that affects the estimation of $P(X_k \cap X_d | C, K)$. We can ignore the event's that is too rare given our
    knowleadge since they won't affect the result too much after multiplying with a probability near 0.

    \section{Description}

    \subsection{The goal}

    \subsection{The method}

    \section{Examples}

    The blacksmith who fail to nail in the horseshoe unleashes a crescendo of consequences: the horse falls, the rider is
    killed , the battle is lost, and the kingdom is conquered.

    \section{Details}

    \bibliographystyle{apalike}%unsrt
    \bibliography{main}

\end{document}
