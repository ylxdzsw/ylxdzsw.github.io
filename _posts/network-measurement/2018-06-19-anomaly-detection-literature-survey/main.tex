\documentclass{manuscript}

\usepackage{textcomp}
\usepackage{enumitem}
\usepackage[utf8]{inputenc}

\title{Literature Survey of Network Anomaly Detection}
\author{Zhang Shiwei}
\date{June 2018}

\begin{document}
    \maketitle

    \section{Paper Review}

    \subsection{Network Fault Diagnosis Using Data Mining Classifiers \textsuperscript{\cite{rozaki_network_2015}}}

    This paper was presented in AIRCC, 2015 by Eleni Rozaki from the Cardiff University.

    The first section describes the FCAPS framework and the position of their contribution under that framework. The
    FCAPS framework stands for fault, configuration, accounting, performance, and security. Their work focus on fault
    diagnosis.

    The second section is the general process of data mining, i.e., data cleaning, section, pattern mining, and knowledge
    representation. They use Weka to perform the mining.

    In the next section several data mining techniques were explained and compaired:
    \begin{description}
        \item[J48 tree] (more commonly known as C4.5). It builds decision trees by maximizing information gain greedly at
                        each node.\textsuperscript{\cite{quinlan_c4.5:_1993}}
        \item[LAD tree] Inducing ADTrees using LogitBoost. An ADTree consists of an alternation of decision nodes, which
                        specify a predicate condition, and prediction nodes, which contain a single number. An instance
                        is classified by an ADTree by following all paths for which all decision nodes are true, and
                        summing any prediction nodes that are traversed.\textsuperscript{\cite{holmes_multiclass_2002}}
        \item[JRip] Alternatively grow and prune rules to build an initial rule set in terms of information gain, Then
                    examine each rule by generate two variants of each rule from randomized data, see which have shorter
                    descrition length.\textsuperscript{\cite{cohen_fast_1995}}
        \item[PART] Generating a decision list by buiding a C4.5 decision tree in each iteration and makes the "best" leaf
                    into a rule. Instances are classified at the first match.\textsuperscript{\cite{frank_generating_1998}}
        \item[Naïve Bayes] Using Bayes rule to calculate the conditional probability with the assumption that all
                           attributes are independent of each other.\textsuperscript{\cite{john_estimating_1995}}
        \item[Bayesnet] Also known as belief networks. It use Bayes rule recursively in a DAG to infer the probabilities
                        of the state of a node.\textsuperscript{\cite{barco_comparison_2006}}
    \end{description}



    \bibliographystyle{unsrt}
    \bibliography{main}
\end{document}
