\documentclass{myarticle}

\usepackage[hidelinks]{hyperref}
\usepackage{textcomp}
\usepackage{amsmath}

\title{A Graph-based Algorithm for Finding Arbitrage Oppotunities in the Cryptocurrency Spot Market}
\author{Zhang Shiwei}
\date{October 2017}

\begin{document}
    \maketitle

    \section{Introduction}

    \subsection{Cryptocurrency Spot Market}

    Cryptocurrencys spot markets are exchanges where people trading their digital coins like Bitcoin, Ethereum, etc.
    On October 19, 2017, the market capitalizations of the top coins are shown in Table \ref{Tab:1}.\footnote{Source:\url{coinmarketcap.com}}

    \begin{table}[hb]
        \centering
        \begin{tabular}{l *{4}{r}}
            Coin & Market Cap. & Price & Volume (24h) \\
            \hline
            Bitcoin & \$102,461,472,454 & \$6158.66 & \$2,708,970,000 \\
            Ethereum & \$28,154,471,164 & \$295.64 & \$451,489,000 \\
            Ripple & \$7,867,600,806 & \$0.20 & \$136,479,000 \\
            Bitcoin Cash & \$5,375,340,170 & \$321.73 & \$158,438,000 \\
            Litecoin & \$3,155,405,470 & \$59.01 & \$176,593,000 \\
            \hline
        \end{tabular}
        \caption{Cryptocurrencys market capitalizations}\label{Tab:1}
    \end{table}

    Many exchanges not only offers trading pairs between digital coins and government-issued currencies, but also supports
    the trading between diffrent coins. This enables some trading strategies like triangular arbitrage.

    Another notable feature of the cryptocurrency spot market is that some exchanges charges no trading fees (e.g. OKEx),
    thus we can trade very rapidly, no matter how small the profit is.

    \subsection{Triangular Arbitrage}\label{triangular arbitrage}

    Suppose we have $x_i$ dollars. If we use them to buy some bitcoins, then use these bitcoins to buy ethereums, and
    finally sell those ethereums for $x_f$ dollars. If $x_f > x_i$, a profit is realized. Since the cryptocurrency spot
    market is active, we can expect $x_f - x_i$ is very small, and only occurs at minute or second time scale.

    Triangular arbitrage opportunities can be identified through the rate product
    \[ \gamma(t) = \prod_{i=1}^{3}r_i(t) \]
    where $r_i(t)$ denotes an exchange rate at time $t$. An arbitrage is theoretically possible if $\gamma > 1$. For each
    group of exchange rates there are two unique rate products that can be calculated. For example, consider the set of
    rates \{BTC/USD, ETH/USD, ETH/BTC\}. If one initially holds US dollars, a possible arbitrage transaction sequence is
    USD\textrightarrow{}BTC\textrightarrow{}ETH\textrightarrow{}USD with the rate product given by
    \[ \gamma_1(t) = \left[\frac{1}{\text{BTC/USD}_{ask}(t)}\right]\cdot
                     \left[\frac{1}{\text{ETH/BTC}_{ask}(t)}\right]\cdot
                     \left[\text{ETH/USD}_{bid}(t)\right] \]
    and the rate product of the other possible trading sequence is
    \[ \gamma_2(t) = \left[\text{BTC/USD}_{bid}(t)\right]\cdot
                     \left[\text{ETH/BTC}_{bid}(t)\right]\cdot
                     \left[\frac{1}{\text{ETH/USD}_{ask}(t)}\right] \]
    These two rate products define all possible arbitrage transactions using this set of trading pairs.\cite{fenn2009mirage}

    \subsection{Inter-market Arbitrage}

    Suppose there are two exchange $E_1$ and $E_2$ offering the same trading pair ETH/BTC with price $ask_i(t)$ and $bid_i(t)$
    at time $t$, where $i = {1,2}$. An inter-market arbitrage strategy can be buying ETH with BTC in $E_1$ and then sell
    them for BTC in $E_2$. If $ask_1(t) < bid_2(t)$ and you managed to trade at both price, a profit $bid_2(t) - ask_1(t)$
    then got realized.

    This is pretty similar with the triangular arbitrage described in section \ref{triangular arbitrage}. In fact, we can
    assume there being two implict trading pair between $\text{BTC}_{E_1} \text{/} \text{BTC}_{E_2}$ with both ask and
    bid price always being 1 (or minus transferring fee rates). The Inter-market arbitrage then become a ``quadrangle arbitrage''
    with all theorems the same as triangular arbitrage except for there being 4 nodes.

    \subsection{Graph Algorithms}

    \section{Coinflow}
    \section{Experiments}
    \section{Disscussion}


    \bibliographystyle{apalike}%unsrt
    \bibliography{main}

\end{document}
