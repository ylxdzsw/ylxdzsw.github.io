\documentclass{manuscript}

\usepackage{textcomp}
\usepackage{enumitem}

\pagenumbering{gobble}

\title{Summary of Neural Adaptive Video Streaming with Pensieve}
\author{Zhang Shiwei}
\date{May 2018}

\begin{document}
    \maketitle

    In \textit{Neural Adaptive Video Streaming with Pensieve}\cite{mao_neural_2017}, Mao et al. presented a novel system
    called \textit{Pensieve} as an alternative to current adaptive bitrate (ABR) algorithms in video streaming. The authors
    also made extensive experiments in both simulated and real environments to show the effectiveness and robustness of it.
    As declared in the paper, Pensieve improves the quality of experience (QoE) for 12\% - 25\% than current state-of-the-art
    algorithms, and generalize well to various types of networks.

    The key idea of Pensieve is employing a reinforcement learning (RL) algorithm to train a neural network model as a
    replacement of current hand-crafted ABR models. RL considers an agent interacts with an environment, where at each
    time step the agent chooses an action and get a reward. By setting network observations and past actions like
    bandwidth, bitrate, and buffer size as the environment, bitrate of the next chunk as the action, and QoE as the reward,
    existing RL algorithms can be used to learn a model that select bitrates automatically to maximize QoE under certain
    observations.

    Accroding to the paper, the primary challenges of ABR algorithms are
    \begin{itemize}[noitemsep, nolistsep]
        \item the variability of network throughput;
        \item the conflicting QoE requirements (high bitrate, minimal rebuffering, smoothness, etc.);
        \item the cascading effects of bitrate decisions (e.g., selecting a high bitrate may drain the playback buffer
        to a dangerous level and cause rebuffering in the future);
        \item and the coarse-grained nature of ABR decisions;
    \end{itemize}

    Current algorithms, as stated in the paper, cannot address these problems well because they rely on fixed heuristics
    or simplified (and thus inaccurate) system models. By contrast, RL-generated algorithms learn from actual performance
    resulting from different decisions. By incorporating this information into a flexible neural network policy, RL-generated
    ABR algorithms can automatically optimize for different network characteristics and QoE objectives.

    The architecture of the neural network is basically a 3-layer MLP, with an additional 1-D convolution layer for
    sequential data like past throughput or next chunk sizes. The authors came up with this architecture by experiments.
    However, for sequential data, recurrent neural networks are often considered performing better than convolutional
    neural networks\cite{lecun_deep_2015}, which is used in this paper.

    I was skeptical about the work because the authors train the model in a simulated environment, while claim it performs
    better than other models that may be based on the statistics of even larger corpus. However, the paper gives a lot
    of experiment details including some in the real environment. The only explanation is that Pensieve did design a
    better generic ABR algorithm from the network trace data than the previous experts. We can try this method in similar
    tasks like TCP congestion control algorithms.

    \bibliographystyle{unsrt}
    \bibliography{main}
\end{document}
